%%%%%%%%%%%%%%%%%%%%%%%%%%%%%%%%%%%%%%%%%
% Medium Length Professional CV
% LaTeX Template
% Version 2.0 (8/5/13)
%
% This template has been downloaded from:
% http://www.LaTeXTemplates.com
%
% Original author:
% Trey Hunner (http://www.treyhunner.com/)
%
% Important note:
% This template requires the resume.cls file to be in the same directory as the
% .tex file. The resume.cls file provides the resume style used for structuring the
% document.
%
%%%%%%%%%%%%%%%%%%%%%%%%%%%%%%%%%%%%%%%%%

%packages
\documentclass[8pt]{resume}
\usepackage[brazilian]{babel}
\usepackage[utf8]{inputenc}
%\usepackage{wasysym}
\usepackage{fontawesome}
\usepackage[misc]{ifsym}
%document margins
\usepackage[left=0.4in,top=0.6in,right=0.4in,bottom=0.6in]{geometry}

%macros
\newcommand{\tit}[1]{\textit{#1}}
\newcommand{\tbf}[1]{\textbf{#1}}
\newcommand{\ttt}[1]{\texttt{#1}}

%doc begin
\name{Erik de Godoy Perillo}
%\address{Rua Paulinia, 396 \\ Americana, SP -- Brazil}
\address{Americana, SP -- Brasil}
\address{\faMobilePhone~+5519996255727
    ~~\Letter~erik.perillo@gmail.com
    ~~$\triangleright$~erikperillo.xyz}

\begin{document}

\begin{rSection}{Experiência}

%\begin{rSubsection}{Phoenix Team of Robotics}{2013 - 2016}{Project Manager}{University of Campinas, Brazil}
\begin{rSubsection}{Líder de Projeto}{2013 - 2016}{Equipe Phoenix de Robótica}{Universidade Estadual de Campinas, SP}
    \item Liderança de dois projetos bem-sucedidos com cerca de seis pessoas
        cada.
    \item Projeto de três robôs autônomos e
        criação de soluções em visão computacional para robôs usando OpenCV.
    \item Liderança do projeto \tit{Piranha} (2015-2016):
        Um mini-veículo autônomo capaz de reconhecimento visual de
        alvos/obstáculos. Primeiro lugar (com recorde) na competição
        latino-americana de robótica da \tit{Robocore 2016}.
    %\item Leader of project \tit{Baleia} (7 people team, 2014-2015):
    %    An autonomous mini-vehicle.
    %    First newcomers in \tit{Robocore}'s latin-american competition to
    %    get to the podium.
\end{rSubsection}

%\begin{rSubsection}{Institute of Computing}{2016 - Present}{Undergraduate Researcher on Artificial Intelligence (grant by CNPQ)}
\begin{rSubsection}{Pesquisador em Inteligência Artificial}{2016 - Presente}{Instituto de Computação (bolsa PIBIC-CNPQ)}
    {Universidade Estadual de Campinas, SP}
    \item Trabalho visando um \tit{framework} para visão em tempo real
        permitindo a robôs explorar ambientes diversos.
    \item Criação de \ttt{att}, uma ferramenta de detecção de saliência
        visual em imagens escrita em \tit{Python}.
        Resultados preliminares do \tit{MIT300 benchmark} sugerem
        desempenho comparável com similares modelos já maduros.
    %\item I am currently implementing state-of-the-art techniques for
    %    saliency detection using deep learning.
\end{rSubsection}

\begin{rSubsection}{Pesquisador em Sistemas de Computação}{2014 - 2015}{Laboratório de Mecânica Computacional (bolsa AMD)}{Universidade Estadual de Campinas, SP}
    \item Análise de múltiplas políticas de gerenciamento de memória para
        sistemas \tit{Linux} com arquitetura \tit{NUMA}.
    \item Design de técnicas para determinação de melhor política para
        cada aplicação via perfilamento de código.
\end{rSubsection}

%\begin{rSubsection}{Pesquisador em Visão Computacional}{2013 - 2014}{Centro de Tecnologia da Informação Renato Archer (bolsa PIBIC-CNPQ)}{Campinas, SP}
%\item Construção de uma ferramenta de rastreamento de robôs móveis usando
%    OpenCV\@. Comparado ao sistema antigo, foi atingido ganho de desempenho
%    próximo a duas vezes com necessidade de \tit{hardware} muito mais simples.
%\end{rSubsection}

\end{rSection}

\begin{rSection}{Educação}

    \begin{rSubsection}{Bacharelado em Ciência/Engenharia da Computação (em progresso)}{2012 - Presente}{Universidade Estadual de Campinas (Unicamp)}{Campinas, SP}
    \item PAD (Professor Assistente) em \tit{Estrutura de Dados}. Ajuda
        na administração de aulas de exercício/laboratório.
    %\item GPA: 3.25 (above 86\% of course group).
    \item Coeficiente de Rendimento: 0.813 (acima de 86\% da turma).
    \item Matérias em Engenharia de Controle e Automação (2012-2014) incluindo:
        Dinâmica, Estática, Sistemas Lineares.
    %\item Unicamp is ranked second best
    %    (overall and in Computer Science) in latin america according to
    %    QS rankings 2016.
\end{rSubsection}

\end{rSection}

\begin{rSection}{Experiência Técnica}
\tbf{Línguas}
\begin{rSubsection}{}{}{}{}
    \item Português: Fluente, Nativo.
    \item Inglês: Fluente.
\end{rSubsection}

\tbf{Tecnologias}
\begin{rSubsection}{}{}{}{}
    \item Linguagens: Python, C++, C, Bash, R, Java, Matlab/Octave, Haskell.
    \item Sistemas/Ferramentas: Linux/Windows, Vim, git, OpenCV.
\end{rSubsection}

\tbf{Projetos}
\begin{rSubsection}{}{}{}{}
\item \ttt{hct}: Contador em tempo real de \tit{hashtags} usando \tit{Twitter
    Streaming API} e \tit{Apache Spark.}
\item \ttt{oarg}: \tit{Parsing} de argumentos de linha de comando para \tit{Python}.
\item \ttt{imsg}: Ferramenta de esteganografia em imagens em \tit{Python}.
\item \ttt{ichat}: Chat de linha de comando TCP em \tit{C++} com
    transferência de arquivos e notificações.
\item \ttt{mlct}: Classificador de texto com \tit{Naive Bayes} para grandes
    volumes de dados usando \tit{Spark Mlib}.
\item \ttt{findr}: Localizador de objetos usando classificadores de imagens
    com OpenCV em \tit{Python/C++}.
\end{rSubsection}
\end{rSection}

\end{document}
