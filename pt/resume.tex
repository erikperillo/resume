%%%%%%%%%%%%%%%%%%%%%%%%%%%%%%%%%%%%%%%%%
% Medium Length Professional CV
% LaTeX Template
% Version 2.0 (8/5/13)
%
% This template has been downloaded from:
% http://www.LaTeXTemplates.com
%
% Original author:
% Trey Hunner (http://www.treyhunner.com/)
%
% Important note:
% This template requires the resume.cls file to be in the same directory as the
% .tex file. The resume.cls file provides the resume style used for structuring the
% document.
%
%%%%%%%%%%%%%%%%%%%%%%%%%%%%%%%%%%%%%%%%%

%packages
\documentclass[8pt]{resume}
\usepackage[brazilian]{babel}
\usepackage[utf8]{inputenc}
%\usepackage{wasysym}
\usepackage{fontawesome}
\usepackage{hyperref}
\usepackage[misc]{ifsym}
%document margins
\usepackage[left=1.2cm,top=0.2in,right=1.4cm,bottom=0.2in]{geometry}

%macros
\newcommand{\tit}[1]{\textit{#1}}
\newcommand{\tbf}[1]{\textbf{#1}}
\newcommand{\ttt}[1]{\texttt{#1}}

%doc begin
\name{Erik de Godoy Perillo}
%\address{Rua Paulinia, 396 \\ Americana, SP -- Brasil}
\address{Americana, SP -- Brasil}
\address{\faMobilePhone~+5519996255727
   ~~\Letter~erik.perillo@gmail.com
   ~~$\triangleright$~erikperillo.xyz}

\begin{document}

\begin{rSection}{Experiência}

%\begin{rSubsection}{Phoenix Team of Robotics}{2013 - 2016}{Project Manager}{Universidade Estadual de Campinas, Brasil}
\begin{rSubsection}{Gerente de Projeto}{2013 - 2016}{Equipe Phoenix de Robótica da Unicamp}{Campinas, Brasil}
    %\item Led two successful teams with around 6 people each.
    \item Líder dos projetos \tit{Baleia/Piranha}, dois mini-veículos
        autônomos (time de 5-7 pessoas).
        Terceiro/primeiro lugar nas competição latino-americanas da
        \tit{Robocore} de 2015/2016, criando um novo recorde na categoria.
    \item Desenvolveu três robôs autônomos, criando sistemas de
        navegação, comunicação e visão computacional.
\end{rSubsection}

%\begin{rSubsection}{Institute of Computing}{2016 - Present}{Undergraduate Researcher on Artificial Intelligence (grant by CNPQ)}
\begin{rSubsection}{Iniciação científica em Inteligência Artificial}{2016 - Presente}{Instituto de Computação, Unicamp}
    {Campinas, Brasil}
    %\item Working towards a framework for real-time vision
    %    allowing mobile robots to explore a variety of environments.
    \item Criou \ttt{att}, um sistema de detecção de saliência visual
        baseado na visão humana com \tit{Deep Learning}.
        %Our preliminary results from MIT300 benchmark suggest performance
        %close to other models with similar architecture.
    \item Prêmio de melhor Iniciação Científica no WTD2017 do IC-Unicamp.
    %\item I am currently implementing state-of-the-art techniques for
    %    saliency detection using deep learning.
\end{rSubsection}

\begin{rSubsection}{Iniciação científica em Computação de Alto Desempenho}{2014 - 2015}{Laboratório de Mecânica Computacional, Unicamp}{Campinas, Brasil}
    %\item Analysed Linux memory management policies for NUMA systems and multiple memory contention metrics.
    \item Criou um modelo preditivo usando Aprendizado de Máquina para inferir
        o desempenho em sistemas ccNUMA,
        salvando $66\%$ do tempo na determinação da melhor política de
        memória para aplicações.
    \item Nosso trabalho resultou em um artigo aceito para a ERAD-SP 2017.
\end{rSubsection}

\begin{rSubsection}{Iniciação Científica em Visão Computacional}{2013 - 2014}{Centro de Technologia da Informação Renato Archer}{Campinas, Brasil}
    \item Construiu ferramenta para rastreamento
        de robôs móveis usando OpenCV (C++).
        Comparado ao sistema antigo, obtivemos o rastreamento de mais
        objetos, sem precisar de hardware especial (apenas webcams).
\end{rSubsection}

\end{rSection}

\begin{rSection}{Educação}

    \begin{rSubsection}{Bacharelado em Ciência/Engenharia da Computação (em progresso)}{2015 - Presente}{Universidade Estadual de Campinas (Unicamp)}{Campinas, Brasil}
    \item Monitor (2016/2017) - Estruturas de Dados.
        Ajudou a elaborar/administrar trabalhos para a matéria.
    \item Coeficiente de Rendimento: 8.49/10 (acima de 92\% da turma).
    \item Fez matérias em Engenharia de Controle e Automação (2012-2014)
        como Dinâmica, Sistemas Lineares.
    %\item Unicamp is ranked second best
    %    (overall and in Computer Science) in latin america according to
    %    QS rankings 2016.
    \end{rSubsection}

\end{rSection}

\begin{rSection}{Idiomas}

    \begin{rSubsection}{}{}{}{}
        \item \tbf{Portugês}: Língua nativa.
        \item \tbf{Inglês}: Proficiente.
    \end{rSubsection}

\end{rSection}


\begin{rSection}{Projetos}

\begin{rSubsection}{}{}{}{}
    \item \ttt{golb}: Plataforma minimalista de blog feita em Django.
    \item \ttt{Robô Piranha}:
        fez o sistema de intra-comunicação com protocolo UDP (C++),
        sistema de visão com CUDA OpenCV (C++/Python),
        ajudou a fazer o sistema de controle PID com plataformas NXP (C/C++).
    \item \ttt{hct}: Contador de \tit{hashtags} em tempo real usando
        \tit{Twitter Streaming API} e Apache Spark.
    \item \ttt{Robô Baleia}:
        fez o sistema de navegação com a biblioteca Adafruit BBIO (Python),
        sistema de visão usando um classificador de imagens para detecção
        de objetos em OpenCV (Python).
    %\item \ttt{ichat}: Chat de linha de comando em TCP (C++) com transferência
    %    de arquivos e notificações.
    %\item \ttt{imsg}: Steganography for images written in Python.
    %\item \ttt{mlct}: Naive Bayes text classifier for big volumes of data using
    %    Apache Spark Mlib.
    %\item \ttt{findr}: Object finder using image classifier with OpenCV in
    %    Python/C++.
\end{rSubsection}

\end{rSection}


\end{document}
