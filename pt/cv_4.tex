%%%%%%%%%%%%%%%%%%%%%%%%%%%%%%%%%%%%%%%%%
% Medium Length Professional CV
% LaTeX Template
% Version 2.0 (8/5/13)
%
% This template has been downloaded from:
% http://www.LaTeXTemplates.com
%
% Original author:
% Trey Hunner (http://www.treyhunner.com/)
%
% Important note:
% This template requires the resume.cls file to be in the same directory as the
% .tex file. The resume.cls file provides the resume style used for structuring the
% document.
%
%%%%%%%%%%%%%%%%%%%%%%%%%%%%%%%%%%%%%%%%%

%----------------------------------------------------------------------------------------
%	PACKAGES AND OTHER DOCUMENT CONFIGURATIONS
%----------------------------------------------------------------------------------------

\documentclass[8pt]{resume} % Use the custom resume.cls style
\usepackage[brazilian]{babel}
\usepackage[utf8]{inputenc}

\newcommand{\cvitem}[2]
    {{\bf #1} \hfill {\em \bf {#2}}\\}
\newcommand{\tit}[1]{\textit{#1}}
\newcommand{\tbf}[1]{\textbf{#1}}
\newcommand{\ttt}[1]{\texttt{#1}}

\usepackage[left=0.4in,top=0.6in,right=0.4in,bottom=0.6in]{geometry} % Document margins

\name{Erik de Godoy Perillo} % Your name
\address{Rua Paulinia, 396 \\ Americana, SP -- Brazil}
\address{Phone: +5519995177505 \\ erik.perillo@gmail.com} % Your phone number and email

\begin{document}

%----------------------------------------------------------------------------------------
%	EDUCATION SECTION
%----------------------------------------------------------------------------------------

%----------------------------------------------------------------------------------------
%	WORK EXPERIENCE SECTION
%----------------------------------------------------------------------------------------

\begin{rSection}{Experience}

\begin{rSubsection}{Phoenix Team of Robotics}{2013 - 2016}{Project Manager}{University of Campinas, Brazil}
    \item Lead two successful projects with around 8 people each.
    \item Helped design three autonomous robots and
        created a Computer Vision framework for robots using OpenCV.
    \item Leader of project \tit{Piranha} (2015-2016):
        An autonomous mini-vehicle with trajectory planning/execution
        and targets/obstacles recognition.
        We got first place, scoring a new record in a latin-american
        robotics challenge.
    \item Leader of project \tit{Baleia} (2014-2015):
        An autonomous mini-vehicle.
        We competed in a latin-american robotics challenge
        being the first newcomers in category to get third place.
\end{rSubsection}

%------------------------------------------------

\begin{rSubsection}{Institute of Computing}{2016 - Present}{Undergraduate Researcher on Artifitial Intelligence (CNPQ)}{University of Campinas, Brazil}
    \item The project aims at developing a general framework for real-time
        objects detection/recognition allowing mobile robots to explore
        unkwown environment.
    \item Created \ttt{att}, a visual saliency detection model written in
        Python that predict where people look on images.
        Our preliminary results suggest competitive performance when compared
        to similar mature models in MIT300 benchmark.
    \item I am currently implementing state-of-the-art techniques for
        saliency detection using deep learning.
\end{rSubsection}

\begin{rSubsection}{Computational Mechanics Laboratory}{2014 - 2015}{Undergraduate Researcher on Computer Systems (AMD)}{University of Campinas, Brazil}
    \item Analysed multiple memory management techniques by Linux Kernel
        for NUMA systems. Designed techiques for determining which policy
        to use via application profiling.
\end{rSubsection}

\begin{rSubsection}{Renato Archer Center of Technology}{2013 - 2014}{Undergraduate Researcher on mobile robotics (CNPQ)}{Campinas, Brazil}
    \item Built a framework for real-time indoors tracking of
        quadrotors using OpenCV. Compared to the older system,
        we achieved real-time tracking of twice as much objects
        and no need for special hardware, only simple webcams.
\end{rSubsection}

%------------------------------------------------

\end{rSection}

%----------------------------------------------------------------------------------------
%	TECHNICAL STRENGTHS SECTION
%----------------------------------------------------------------------------------------
\begin{rSection}{Education}

\begin{rSubsection}{University of Campinas (Unicamp), Brazil}{2014 - Present}
    {B.S. in Computer Science/Engineering}{}
    %\item B.S. in Computer Science/Engineering.
    \item T.A. (2016) in \tit{Algorithms and data Structures 2}. I helped
        teach and design/administer exercises sessions.
    \item Overall GPA: 3.4 (above 90\% of class).
    \item Unicamp is ranked second best
        (overall and in Computer Science) in latin america according to
        QS rankings 2016.
\end{itemize}

\end{rSection}


\begin{rSection}{Additional Information}

\begin{rSubsection}{}{}{}{}
    \item Computer Languages: Python, C++, C, Java, shell-scripting, Haskell,
        Scheme.
    \item Interest in Machine Learning/Computer Vision:
        built object classifiers with OpenCV,
        text classifiers for big volumes of data using Apache Spark.
    \item A wide variety of small projects, such as a TCP/IP chat in C++,
        Steganography in images with Python, a real time hashtag counter
        using Twitter streaming API and Apache Spark.
\end{tabular}

\end{rSection}

%----------------------------------------------------------------------------------------
%	EXAMPLE SECTION
%----------------------------------------------------------------------------------------

%\begin{rSection}{Section Name}

%Section content\ldots

%\end{rSection}

%----------------------------------------------------------------------------------------

\end{document}
