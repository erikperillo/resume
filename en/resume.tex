%%%%%%%%%%%%%%%%%%%%%%%%%%%%%%%%%%%%%%%%%
% Medium Length Professional CV
% LaTeX Template
% Version 2.0 (8/5/13)
%
% This template has been downloaded from:
% http://www.LaTeXTemplates.com
%
% Original author:
% Trey Hunner (http://www.treyhunner.com/)
%
% Important note:
% This template requires the resume.cls file to be in the same directory as the
% .tex file. The resume.cls file provides the resume style used for structuring the
% document.
%
%%%%%%%%%%%%%%%%%%%%%%%%%%%%%%%%%%%%%%%%%

%packages
\documentclass[8pt]{resume}
\usepackage[brazilian]{babel}
\usepackage[utf8]{inputenc}
%\usepackage{wasysym}
\usepackage{fontawesome}
\usepackage{hyperref}
\usepackage[misc]{ifsym}
%document margins
\usepackage[left=1.4cm,top=0.3in,right=1.6cm,bottom=0.5in]{geometry}

%macros
\newcommand{\tit}[1]{\textit{#1}}
\newcommand{\tbf}[1]{\textbf{#1}}
\newcommand{\ttt}[1]{\texttt{#1}}

%doc begin
\name{Erik de Godoy Perillo}
%\address{Rua Paulinia, 396 \\ Americana, SP -- Brazil}
\address{Americana, SP -- Brazil}
\address{\faMobilePhone~+55 31 973248225
   ~~\Letter~erik.perillo@gmail.com
   ~~$\triangleright$~erikperillo.xyz}

\begin{document}

\begin{rSection}{Experience}

\begin{rSubsection}{Agronow}{2018 - Present}{Machine Learning Engineer}
    {São Paulo, Brazil}
\item Built Machine Learning solutions for agricultural crop identification
        using public satellite imagery.
\end{rSubsection}

\begin{rSubsection}{Google}{2017 - 2018}{Software Engineering Intern}
    {Belo Horizonte, Brazil}
    \item Worked on the sports experience in search, successfully developing/launching a new feature to production.
\end{rSubsection}

\begin{rSubsection}{GAIIA tech (startup)}{2017 - 2017}{Co-founder, CTO}
    {São Paulo, Brazil}
    \item Leader of the development team, providing Deep Learning solutions for agriculture.
\end{rSubsection}

%\begin{rSubsection}{Phoenix Team of Robotics}{2013 - 2016}{Project Manager}{University of Campinas, Brazil}
\begin{rSubsection}{Phoenix Robotics Team - Unicamp}{2013 - 2016}{Project Manager}
    {Campinas, Brazil}
    %\item Led two successful teams with around 6 people each.
\item Leader of projects \tit{Piranha}/\tit{Baleia},
        two autonomous mini-vehicles (5-7 people team).
        Third/First place in \tit{Robocore}'s latin-american 2015/2016 robotics
        challenge, setting a new record.
    \item Built navigation, communication and computer vision systems
        during the conception of 3 autonomous robots.
\end{rSubsection}

%\begin{rSubsection}{Institute of Computing}{2016 - Present}{Undergraduate Researcher on Artificial Intelligence (grant by CNPQ)}
\begin{rSubsection}{Institute of Computing - Unicamp}{2016 - 2017}{Undergraduate Researcher on Deep Learning}
    {Campinas, Brazil}
    %\item Working towards a framework for real-time vision
    %    allowing mobile robots to explore a variety of environments.
    \item Created \tit{DeepPeek}, a Convolutional Neural Network for visual saliency detection.
        Our model has around 3/4 less parameters than similar methods yet
        achieved top-10 performance on MIT300 benchmark.
    %\item Best Undergraduate Research Project Award on WTD2017 conference at
    %    Unicamp.
        %Our preliminary results from MIT300 benchmark suggest performance
        %close to other models with similar architecture.
    %\item I am currently implementing state-of-the-art techniques for
    %    saliency detection using deep learning.
\end{rSubsection}
\end{rSection}

\begin{rSection}{Education}
\begin{rSubsection}{Master of Science (MS), Computer Science}{2018 - Present}{University of Campinas (Unicamp)}{Campinas, Brazil}
    \item Ranked first place in the admission process.
\end{rSubsection}

\begin{rSubsection}{Bachelor of Science (BS), Computer Science (graduated with distinction)}{2015 - 2018}{University of Campinas (Unicamp)}{Campinas, Brazil}
    \item Teaching Assistant (2016/2017) in \tit{Data Structures}. Helped
        design/administer programming assignments.
    \item Last GPA: 8.8/10 (first in class).
    \item Three Research projects on Computer Vision, High Performance Computing and Deep Learning.
    %\item Coursework in Control Engineering (2012-2014) including:
    %    Dynamics, Statics, Linear Systems.
    %\item Unicamp is ranked second best
    %    (overall and in Computer Science) in latin america according to
    %    QS rankings 2016.
\end{rSubsection}
\end{rSection}

\begin{rSection}{Awards}
\begin{rSubsection}{}{}{}{}
    \vspace{-0.5em}
    \item \tbf{Best Undergraduate Research Project} for the work presented at \tit{WTD2017} conference at Unicamp.
    \item \tbf{Alumni Scholarship}: awarded to 4 selected students for their undergraduate research projects in 2017.
\end{rSubsection}
\end{rSection}

\begin{rSection}{Other Projects}
\begin{rSubsection}{}{}{}{}
    \vspace{-0.5em}
    \item \tbf{\ttt{golb}}: Minimalistic blog platform built with Django.
    %\item \tbf{\ttt{Piranha Robot}}:
    %    built inter-communication system using UDP protocol (C++),
    %    vision system using CUDA OpenCV (C++/Python),
    %    helped build PID control unit using NXP platform/sensors (C/C++).
    %\item \tbf{\ttt{hct}}: Real time hashtag counter using Twitter Streaming API and
    %    Apache Spark.
    %\item \tbf{\ttt{Baleia Robot}}:
    %    built navigation system with Adafruit's BBIO library and
    %    vision system with OpenCV.
    %\item \tbf{\ttt{oarg}}: A command-line argument parser for Python.
    \item \tbf{\ttt{ichat}}: TCP command-line chat in C++ with file transfer and
        notifications.
    %\item \ttt{imsg}: Steganography for images written in Python.
    %\item \ttt{mlct}: Naive Bayes text classifier for big volumes of data using
    %    Apache Spark Mlib.
    %\item \ttt{findr}: Object finder using image classifier with OpenCV in
    %    Python/C++.
\end{rSubsection}

\end{rSection}

\end{document}
