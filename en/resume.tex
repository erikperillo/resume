%%%%%%%%%%%%%%%%%%%%%%%%%%%%%%%%%%%%%%%%%
% Medium Length Professional CV
% LaTeX Template
% Version 2.0 (8/5/13)
%
% This template has been downloaded from:
% http://www.LaTeXTemplates.com
%
% Original author:
% Trey Hunner (http://www.treyhunner.com/)
%
% Important note:
% This template requires the resume.cls file to be in the same directory as the
% .tex file. The resume.cls file provides the resume style used for structuring the
% document.
%
%%%%%%%%%%%%%%%%%%%%%%%%%%%%%%%%%%%%%%%%%

%packages
\documentclass[8pt]{resume}
\usepackage[brazilian]{babel}
\usepackage[utf8]{inputenc}
%\usepackage{wasysym}
\usepackage{fontawesome}
\usepackage{hyperref}
\usepackage[misc]{ifsym}
%document margins
\usepackage[left=1.4cm,top=0.5in,right=1.6cm,bottom=0.5in]{geometry}

%macros
\newcommand{\tit}[1]{\textit{#1}}
\newcommand{\tbf}[1]{\textbf{#1}}
\newcommand{\ttt}[1]{\texttt{#1}}

%doc begin
\name{Erik de Godoy Perillo}
%\address{Rua Paulinia, 396 \\ Americana, SP -- Brazil}
\address{Americana, SP -- Brazil}
\address{\faMobilePhone~+5519996255727
   ~~\Letter~erik.perillo@gmail.com
   ~~$\triangleright$~erikperillo.xyz}

\begin{document}

\begin{rSection}{Experience}

%\begin{rSubsection}{Phoenix Team of Robotics}{2013 - 2016}{Project Manager}{University of Campinas, Brazil}
\begin{rSubsection}{Project Manager}{2013 - 2016}{Phoenix Robotics Team of Unicamp}{Campinas, Brazil}
    %\item Led two successful teams with around 6 people each.
\item Leader of projects \tit{Piranha}/\tit{Baleia},
        two autonomous mini-vehicles (5-7 people team).
        Third/First place in \tit{Robocore}'s latin-american 2015/2016 robotics
        challenge, setting a new record.
    \item Helped create three autonomous robots, building navigation,
        communication and computer vision systems.
\end{rSubsection}

%\begin{rSubsection}{Institute of Computing}{2016 - Present}{Undergraduate Researcher on Artificial Intelligence (grant by CNPQ)}
\begin{rSubsection}{Undergraduate Researcher on Artificial Intelligence}{2016 - Present}{Institute of Computing, University of Campinas}
    {Campinas, Brazil}
    %\item Working towards a framework for real-time vision
    %    allowing mobile robots to explore a variety of environments.
    \item Created a visual saliency detection system using Deep Learning.
        Our model has around 3/4 less parameters than similar methods yet
        achieves top-10 performance on MIT300 benchmark.
        %Our preliminary results from MIT300 benchmark suggest performance
        %close to other models with similar architecture.
    %\item I am currently implementing state-of-the-art techniques for
    %    saliency detection using deep learning.
\end{rSubsection}

\begin{rSubsection}{Undergraduate Researcher on High Performance Computing}{2014 - 2015}{Computational Mechanics Laboratory, University of Campinas}{Campinas, Brazil}
    \item Designed a tool using Machine Learning to infer performance
        in ccNUMA systems, saving $66\%$ of the time to
        determine the best memory policy for applications.
    \item Our work resulted in a paper accepted for ERAD-SP 2017 conference in Brazil.
\end{rSubsection}

\begin{rSubsection}{Undergraduate Researcher on Computer Vision}{2013 - 2014}{Renato Archer Center of Technology}{Campinas, Brazil}
    \item Built a tool for real-time indoors tracking of
        mobile robots using OpenCV (C++),
        achieving tracking of more objects compared to the older system
        and no need for special hardware (only webcams).
\end{rSubsection}

\end{rSection}

\begin{rSection}{Education}

    \begin{rSubsection}{B.S. in Computer Science (in progress)}{2015 - Present}{University of Campinas (Unicamp)}{Campinas, Brazil}
    \item Teaching Assistant (2016/2017) in \tit{Data Structures}. Helped
        design/administer programming assignments.
    \item GPA: 8.49/10 (above 92\% of class).
    \item Coursework in Control Engineering (2012-2014) including:
        Dynamics, Statics, Linear Systems.
    %\item Unicamp is ranked second best
    %    (overall and in Computer Science) in latin america according to
    %    QS rankings 2016.
\end{rSubsection}

\end{rSection}

\begin{rSection}{Projects}
\begin{rSubsection}{}{}{}{}
    \vspace{-0.5em}
    \item \tbf{\ttt{golb}}: Minimalistic blog platform built with Django.
    \item \tbf{\ttt{Piranha Robot}}:
        built inter-communication system using UDP protocol (C++),
        vision system using CUDA OpenCV (C++/Python),
        helped build PID control unit using NXP platform/sensors (C/C++).
    \item \tbf{\ttt{hct}}: Real time hashtag counter using Twitter Streaming API and
        Apache Spark.
    \item \tbf{\ttt{Baleia Robot}}:
        built navigation system with Adafruit's BBIO library and
        vision system with OpenCV.
    \item \tbf{\ttt{oarg}}: A command-line argument parser for Python.
    \item \tbf{\ttt{ichat}}: TCP command-line chat in C++ with file transfer and
        notifications.
    %\item \ttt{imsg}: Steganography for images written in Python.
    %\item \ttt{mlct}: Naive Bayes text classifier for big volumes of data using
    %    Apache Spark Mlib.
    %\item \ttt{findr}: Object finder using image classifier with OpenCV in
    %    Python/C++.
\end{rSubsection}

\end{rSection}

\end{document}
