%%%%%%%%%%%%%%%%%%%%%%%%%%%%%%%%%%%%%%%%%
% Medium Length Professional CV
% LaTeX Template
% Version 2.0 (8/5/13)
%
% This template has been downloaded from:
% http://www.LaTeXTemplates.com
%
% Original author:
% Trey Hunner (http://www.treyhunner.com/)
%
% Important note:
% This template requires the resume.cls file to be in the same directory as the
% .tex file. The resume.cls file provides the resume style used for structuring the
% document.
%
%%%%%%%%%%%%%%%%%%%%%%%%%%%%%%%%%%%%%%%%%

%packages
\documentclass[8pt]{resume}
\usepackage[brazilian]{babel}
\usepackage[utf8]{inputenc}
%\usepackage{wasysym}
\usepackage{fontawesome}
\usepackage{hyperref}
\usepackage[misc]{ifsym}
%document margins
\usepackage[left=0.4in,top=0.6in,right=0.4in,bottom=0.6in]{geometry}

%macros
\newcommand{\tit}[1]{\textit{#1}}
\newcommand{\tbf}[1]{\textbf{#1}}
\newcommand{\ttt}[1]{\texttt{#1}}

%doc begin
\name{Erik de Godoy Perillo}
%\address{Rua Paulinia, 396 \\ Americana, SP -- Brazil}
\address{Americana, SP -- Brazil}
\address{\faMobilePhone~+5519996255727
   ~~\Letter~erik.perillo@gmail.com
   ~~$\triangleright$~erikperillo.xyz}

\begin{document}

\begin{rSection}{Experience}

%\begin{rSubsection}{Phoenix Team of Robotics}{2013 - 2016}{Project Manager}{University of Campinas, Brazil}
\begin{rSubsection}{Project Manager}{2013 - 2016}{Phoenix Team of Robotics}{University of Campinas, Brazil}
    %\item Led two successful teams with around 6 people each.
    \item Leader of project \tit{Piranha} (5 people team, 2015-2016):
        An autonomous mini-vehicle.
        First place in \tit{Robocore}'s latin-american 2016 robotics challenge, setting a new record.
    \item Helped design three autonomous robots, creating navigation,
        communication and computer vision systems.
    %\item Leader of project \tit{Baleia} (7 people team, 2014-2015):
    %    An autonomous mini-vehicle.
    %    First newcomers in \tit{Robocore}'s latin-american competition to
    %    get to the podium.
\end{rSubsection}

%\begin{rSubsection}{Institute of Computing}{2016 - Present}{Undergraduate Researcher on Artificial Intelligence (grant by CNPQ)}
\begin{rSubsection}{Undergraduate Researcher on Artificial Intelligence}{2016 - Present}{Institute of Computing (grant by CNPQ)}
    {University of Campinas, Brazil}
    \item Working towards a framework for real-time vision
        allowing mobile robots to explore a variety of environments.
    \item Created \ttt{att}, a visual saliency detection model written in
        Python for predicting where people look on images.
        Our preliminary results from MIT300 benchmark suggest performance
        close to similar mature models.
    %\item I am currently implementing state-of-the-art techniques for
    %    saliency detection using deep learning.
\end{rSubsection}

\begin{rSubsection}{Undergraduate Researcher on Computer Systems}{2014 - 2015}{Computational Mechanics Laboratory (grant by AMD)}{University of Campinas, Brazil}
    \item Analysed Linux memory management policies for NUMA systems and multiple memory contention metrics.
    \item Designed a predictive model to infer the performance of applications under different memory policies.
\end{rSubsection}

\begin{rSubsection}{Undergraduate Researcher on Computer Vision}{2013 - 2014}{Renato Archer Center of Technology (grant by CNPQ)}{Campinas, Brazil}
    \item Built a tool for real-time indoors tracking of
        mobile robots using OpenCV\@. Compared to the older system,
        we achieved real-time tracking of twice as much objects
        and no need for special hardware (only simple webcams).
\end{rSubsection}

\end{rSection}

\begin{rSection}{Education}

    \begin{rSubsection}{B.S. in Computer Science/Engineering (in progress)}{2012 - Present}{University of Campinas (Unicamp)}{Campinas, Brazil}
    \item T.A. (2016/2017) in \tit{Data Structures}. Helped
        teach and design/administer exercises sessions.
    \item GPA: 8.49/10 (above 92\% of class).
    \item Coursework in Control Engineering (2012-2014) including:
        Dynamics, Statics, Linear Systems.
    %\item Unicamp is ranked second best
    %    (overall and in Computer Science) in latin america according to
    %    QS rankings 2016.
\end{rSubsection}

\end{rSection}

\begin{rSection}{Projects}

\begin{rSubsection}{}{}{}{}
    \item \ttt{golb}: Minimalistic blog platform built with Django.
    \item \ttt{Piranha Robot}:
        built inter-communication system using UDP protocol (C++),
        vision system using CUDA OpenCV (C++/Python),
        helped build PID control unit using NXP platform/sensors (C/C++),
    \item \ttt{hct}: Real time hashtag counter using Twitter Streaming API and
        Apache Spark.
    \item \ttt{Baleia Robot}:
        built navigation system using Adafruit's BBIO library (Python),
        vision system using an image classifier to detect objects in OpenCV (Python).
    \item \ttt{oarg}: A command-line argument parser for Python.
    \item \ttt{ichat}: TCP command-line chat in C++ with file transfer and
        notifications.
    \item \ttt{imsg}: Steganography for images written in Python.
    %\item \ttt{mlct}: Naive Bayes text classifier for big volumes of data using
    %    Apache Spark Mlib.
    %\item \ttt{findr}: Object finder using image classifier with OpenCV in
    %    Python/C++.
\end{rSubsection}

\end{document}
